%%%%%%%%%%%%%%%%%%%%%%%%%%%%%%%%%%%%%%%%%%%%%%%%%%%%%%%%%%%%%%%%
%%         Guide de Programmation Linux pour Robopoly	      %%
%%                       Andrew Watson                        %%
%%                           MT-BA6                           %%
%%%%%%%%%%%%%%%%%%%%%%%%%%%%%%%%%%%%%%%%%%%%%%%%%%%%%%%%%%%%%%%%

%\documentclass[a4paper,draft]{article}
\documentclass[a4paper]{article}
% NB : enlever ``draft'' pour avoir les images!

\usepackage[french]{babel}
\usepackage[utf8]{inputenc}
\usepackage[T1]{fontenc} % for hyphenation of accentuated words
\usepackage{ae,aecompl}  % alternative to the fugly T1 fonts
\usepackage[pdftex]{graphicx} % modern graphics package
\usepackage{multirow}   % multiple rows in tables

%\usepackage{microtype}
%\usepackage{hyph-utf8}

\usepackage{wasysym}    % extended symbols
\usepackage{includes/mystyle}    % added styling
\usepackage{marvosym}   % euro symbol, among other things.
\usepackage[eurosym]{eurofont}

\usepackage[usenames,dvipsnames]{color}  % if using PDFLaTeX
%\usepackage[usenames,dvips]{color}      % for regular LaTeX 

\usepackage{fancyhdr}   % fancy headers
\pagestyle{fancy}

%\usepackage{textcomp}  % some additionnal symbols
\usepackage{units}      % pretty units : \unit[val]{dim}
\usepackage{url}        % pretty urls
\usepackage{listings}   % pretty code listings \begin{lstlisting}

\graphicspath{{.}{./images/}}
\usepackage{subfigure} % multiple figures

%\addtocounter{secnumdepth}{1}  % number paragraphs

\usepackage{lscape}     % use landscape mode with \begin{landscape}
\usepackage{wrapfig}    % invoke Satan and wrap figures in text

\begin{document}
\begin{titlepage}
\nocite{*}      % to make sure bibliography appears in the correct order
  \begin{center}
     
     
    % Upper part of the page
    %\includegraphics[width=4cm]{logo_epfl}\\[0.5cm]
     
    \vfill 
    \textsc{\LARGE Robopoly }\\[1.0cm]

    { \huge \bfseries Guide de programmation Linux}\\[0.4cm]
    \includegraphics[width=12cm]{images/robo_bobo_blanc_big}\\[0.5cm]

    \vfill

     
    % Bottom of the page
    Andrew \textsc{Watson}\\[0.5cm] 
    Version 0.1.1\\[0.5cm]
    {\large \today} 
     
  \end{center}

\end{titlepage}
%\maketitle

\newpage{}

%\fancyhead{}
\fancyfoot{}
\lhead{}
\cfoot{\thepage}        % numéro de page..
\lfoot{Guide Linux pour Robopoly}
%\rfoot{\today}
\rfoot{\today} %% TODO : fix the date

%\begin{abstract}
%\end{abstract}

% CUSTOM COMMANDS FOR THIS REPORT
\newcommand{\expr}[1]{\og \emph{{#1}} \fg} % \expr{word} => « Word »
\newcommand{\important}[1]{\textbf{#1}}
\newcommand{\adchip}[0]{\emph{ADE7169}}


%\renewcommand\contentsname{Plan}  % Rename ``Table des Matières''
\tableofcontents{}

\newpage
\lstset{language=C}
\lstset{language=bash}

\section*{Introduction}
\addcontentsline{toc}{section}{Introduction}
\markboth{Introduction}{\MakeUppercase{Introduction}}
Ce guide présuppose que vous êtes relativement à l'aise avec la ligne de
commande Linux et que vous avez déjà un éditeur de texte préféré. Les outils
décrits ici ne sont pas aussi intuitifs que AVR Studio sous Windows, mais les
résultats seront identiques une fois que tout est configuré correctement.

% blah
Atmel tools only available under Windows (AVR Studio)

Install avr-gcc toolchain for linux

\section{Basic setup}
Pour commencer, il nous faut une suite d'outils pour nous permettre de compiler
le code pour le microcontrôleur AVR. Ceux-ci sont disponibles avec le
gestionnaire de packets de la plupart des distributions Linux.
Les outils nécessaires sont :
\begin{itemize}
  \item avr-libc, un assortiment de librairies pour les fonctions habituelles du
    microcontrôleur
  \item avr-gcc, une extension de GCC pour les microcontrôleurs AVR
\end{itemize}

Exemple pour Ubuntu:
\begin{lstlisting}
  sudo apt-get install binutils gcc-avr avr-libc
\end{lstlisting}
Et pour Arch:
\begin{lstlisting}
  sudo pacman -S gcc-avr binutils-avr avr-libc
\end{lstlisting}
Si le gestionnaire de paquets propose des programmes supplémentaires, c'est
probablement une bonne idée de les installer aussi.

De plus, pour certains des scripts fournis dans cette archive (notamment pour
pouvoir charger les programmes sur le microcontrôleur), il vous faudra installer
python, s'il n'est pas déjà inclus dans votre installation de base :

\begin{lstlisting}
  sudo apt-get install python2.6
\end{lstlisting}

\section{Librairies Robopoly}
Blablabla librairies robopoly qui sont dans cette archive/dans robopolibs
whatever

\section{Chargement des programmes}
Pour charger les programmes sur le microcontrôleurs, nous allons utiliser le
script pygaload qui se trouve dans cette archive. La commande de base pour
charger le programme ``prog.hex'' sur le PRisme est la suivante (à lancer depuis
le dossier dans lequel se trouve pygaload.py):
\begin{lstlisting}
  ./pygaload.py prog.hex -p /dev/ttyUSB0 -V
\end{lstlisting}
Où /dev/ttyUSB0 est le port sur lequel se trouve le programmateur USB.

Pour faciliter le téléchargement des programmes à l'avenir, il y a deux choses
que l'on peut faire. Tout d'abord, ajouter pygaload au \$PATH pour pouvoir
l'appeler depuis n'importe où.

\begin{lstlisting}
  cp pygaload.py /usr/local/bin/pygaload
\end{lstlisting}

On peut maintenant appeler pygaload depuis n'importe quel dossier en tapant
simplement ``pygaload'' au lieu de ``./pygaload.py''.

Pour faciliter les choses encore plus, la meilleure option est d'ajouter une
entrée au Makefile qui appelle automatiquement pygaload avec les bons arguments
pour chaque projet. Le Makefile fourni avec le projet d'exemple contient déjà
une commande ``flash'' qui utilise pygaload pour charger le programme sur le
microcontrôleur. On peut donc charger le programme en invocant la commande
``make flash'' depuis le dossier du projet.

\section{Makefile}
% TODO
blablabla





\end{document}
