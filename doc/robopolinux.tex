%%%%%%%%%%%%%%%%%%%%%%%%%%%%%%%%%%%%%%%%%%%%%%%%%%%%%%%%%%%%%%%%
%%         Guide de Programmation Linux pour Robopoly	      %%
%%                       Andrew Watson                        %%
%%                           MT-BA6                           %%
%%%%%%%%%%%%%%%%%%%%%%%%%%%%%%%%%%%%%%%%%%%%%%%%%%%%%%%%%%%%%%%%

%\documentclass[a4paper,draft]{article}
\documentclass[a4paper]{article}
% NB : enlever ``draft'' pour avoir les images!

\usepackage[french]{babel}
\usepackage[utf8]{inputenc}
\usepackage[T1]{fontenc} % for hyphenation of accentuated words
\usepackage{ae,aecompl}  % alternative to the fugly T1 fonts
\usepackage[pdftex]{graphicx} % modern graphics package
\usepackage{multirow}   % multiple rows in tables

%\usepackage{microtype}
%\usepackage{hyph-utf8}

\usepackage{wasysym}    % extended symbols
\usepackage{includes/mystyle}    % added styling
\usepackage{marvosym}   % euro symbol, among other things.
\usepackage[eurosym]{eurofont}

\usepackage[usenames,dvipsnames]{color}  % if using PDFLaTeX
%\usepackage[usenames,dvips]{color}      % for regular LaTeX 

\usepackage{fancyhdr}   % fancy headers
\pagestyle{fancy}

%\usepackage{textcomp}  % some additionnal symbols
\usepackage{units}      % pretty units : \unit[val]{dim}
\usepackage{url}        % pretty urls
\usepackage{listings}   % pretty code listings \begin{lstlisting}
\usepackage{tikz} % pic/graph package. used for rounded corners :)

\graphicspath{{.}{./images/}}
\usepackage{subfigure} % multiple figures

%\addtocounter{secnumdepth}{1}  % number paragraphs

\usepackage{lscape}     % use landscape mode with \begin{landscape}
\usepackage{wrapfig}    % invoke Satan and wrap figures in text

\usepackage[colorlinks=true]{hyperref}

\hypersetup{
    bookmarks=true,         % show bookmarks bar?
    unicode=true,          % non-Latin characters in Acrobat’s bookmarks
    pdftoolbar=true,        % show Acrobat’s toolbar?
    pdfmenubar=true,        % show Acrobat’s menu?
    pdffitwindow=true,     % window fit to page when opened
    pdfstartview={FitV},    % fits the width of the page to the window
    pdftitle={Robopoly - Guide de Programmation Linux},    % title
    pdfauthor={Andrew Watson},     % author
    pdfsubject={Robopoly Linux Librairies},   % subject of the document
    pdfcreator={Andrew Watson},   % creator of the document
    pdfproducer={Robopoly}, % producer of the document
    pdfkeywords={Robopoly, Linux, AVR}, % list of keywords
    pdfnewwindow=true,      % links in new window
    colorlinks=true,       % false: boxed links; true: colored links
    linkcolor=MidnightBlue,          % color of internal links
    citecolor=RoyalPurple,        % color of links to bibliography
    filecolor=magenta,      % color of file links
    urlcolor=DarkOrchid           % color of external links
}

\begin{document}
\begin{titlepage}
\nocite{*}      % to make sure bibliography appears in the correct order
  \begin{center}
     
     
    % Upper part of the page
    %\includegraphics[width=4cm]{logo_epfl}\\[0.5cm]
     
    \vfill 
    \textsc{\LARGE Robopoly }\\[1.0cm]

    { \huge \bfseries Guide de programmation Linux}\\[0.4cm]
    \includegraphics[width=12cm]{images/robo_bobo_blanc_big}\\[0.5cm]

    \vfill

     
    % Bottom of the page
    Andrew \textsc{Watson}\\[0.5cm] 
    Version 0.1.1\\[0.5cm]
    {\large \today} 
     
  \end{center}

\end{titlepage}
%\maketitle

\newpage{}

%\fancyhead{}
\fancyfoot{}
\lhead{}
\cfoot{\thepage}        % numéro de page..
\lfoot{Guide Linux pour Robopoly}
%\rfoot{\today}
\rfoot{\today} %% TODO : fix the date

%\begin{abstract}
%\end{abstract}

% CUSTOM COMMANDS FOR THIS REPORT
\newcommand{\expr}[1]{\og \emph{{#1}} \fg} % \expr{word} => « Word »
\newcommand{\important}[1]{\textbf{#1}}
\newcommand{\code}[1]{\texttt{#1}}

\newcommand{\urlrobopoly}{\url{http://robopoly.epfl.ch}}
\newcommand{\urldownloads}{\url{http://github.com/tunebird/robopoly-linux/downloads}}
\newcommand{\urlreadme}{
\url{http://github.com/tunebird/robopoly-linux/blob/master/example/README.md}}
\newcommand{\urlgithub}{\url{http://github.com/tunebird/robopoly-linux}}




%\renewcommand\contentsname{Plan}  % Rename ``Table des Matières''
\tableofcontents{}

\newpage

%%%% BEGIN : LSTLISTINGS CONFIG %%%%%%
%%%% TODO : MOVE TO SEPARATE FILE ONCE FINISHED %%%%%
%% see http://www.jorgemarsal.com/blog/2009/06/08/source-code-snippets-in-latex/
\lstset{language=C}
%\definecolor{lightgrey}{RGB}{200,200,200}
\definecolor{grey92}{gray}{0.92}
\definecolor{grey75}{gray}{0.75}
\definecolor{grey45}{gray}{0.45}
\lstdefinestyle{console}
{
  numbers=none,
  basicstyle=\bf\ttfamily,
  backgroundcolor=\color{grey92},
  frame=lrtb,
  framerule=0.5pt,
  linewidth=\textwidth,
}
\lstdefinestyle{avr-c}
{
  style=console
}
\lstdefinestyle{avr-asm}
{
  style=console
}

\lstset{
  style=console
}

%%%%%%% END : LSTLISTINGS CONFIG %%%%%%%%


\section*{Introduction}
\addcontentsline{toc}{section}{Introduction}
\markboth{Introduction}{\MakeUppercase{Introduction}}
Ce guide présuppose que vous êtes relativement à l'aise avec la ligne de
commande Linux et que vous avez déjà un éditeur de texte préféré. Les outils
décrits ici ne sont pas aussi intuitifs que AVR Studio sous Windows, mais les
résultats seront identiques une fois que tout est configuré correctement.

% blah
Atmel tools only available under Windows (AVR Studio)

Install avr-gcc toolchain for linux

\section{Installation des outils}
Pour commencer, il nous faut une suite d'outils pour nous permettre de compiler
le code pour le microcontrôleur AVR. Ceux-ci sont disponibles avec le
gestionnaire de packets de la plupart des distributions Linux.
Les outils nécessaires sont :
\begin{itemize}
  \item avr-libc, un assortiment de librairies pour les fonctions habituelles du
    microcontrôleur
  \item avr-gcc, une extension de GCC pour les microcontrôleurs AVR
\end{itemize}

Exemple pour Ubuntu:
\begin{lstlisting}[style=console]
  sudo apt-get install binutils gcc-avr avr-libc
\end{lstlisting}
Et pour Arch:
\begin{lstlisting}[style=console]
  sudo pacman -S gcc-avr binutils-avr avr-libc
\end{lstlisting}
Si le gestionnaire de paquets propose des programmes supplémentaires, c'est
probablement une bonne idée de les installer aussi.

De plus, pour certains des scripts fournis dans cette archive (notamment pour
pouvoir charger les programmes sur le microcontrôleur), il vous faudra installer
python, s'il n'est pas déjà inclus dans votre installation de base :

\begin{lstlisting}[style=console]
  sudo apt-get install python2.6
\end{lstlisting}

\section{Librairies Robopoly}
Les librairies sont disponibles sur le site de
Robopoly\footnote{\urlrobopoly{}}. La version
(stable) la plus récente se trouve généralement sur
Github\footnote{\urldownloads{}}.

Téléchargez l'archive dans le dossier que vous allez utiliser pour écrire vos
programmes. Dans les exemples suivants, remplacez \code{~/robopoly/} par le
dossier de votre choix.

Commencez par extraire l'archive:
\begin{lstlisting}[style=console]
  cd ~/robopoly
  tar xvzf robopolibs*.tar.gz
  cd robopolibs*
\end{lstlisting}

Ce dossier contient les fichiers suivants:
\begin{itemize}
  \item \code{README.md} Un guide de démarrage rapide, aussi disponible en
    ligne\footnote{\urlreadme{}}.
  \item \code{pygaload.py} Un script python qui permettra de charger les
    programmes sur le microcontrôleur.
  \item \code{robopoly.*} Les librairies standard robopoly
  \item \code{lcam*} La librairie pour la caméra linéaire
  \item \code{Makefile} Pour pouvoir compiler et télécharger les programmes
    facilement
\end{itemize}

\subsection{Premier test}
Avant d'aller plus loin, vous pouvez essayer de compiler le projet d'exemple.
Depuis le dossier que vous venez d'extraire, lancez la commande suivante:
\begin{lstlisting}[style=console]
  make hex 
\end{lstlisting}
Si la compilation s'effectue sans erreurs, vous devriez maintenant avoir un
fichier \code{exemple.hex} dans le dossier actuel. Celui-ci contient le code à
charger sur le microcontrôleur.

\section{Chargement des programmes}
Pour charger les programmes sur le microcontrôleurs, nous allons utiliser le
script pygaload qui se trouve dans cette archive. La commande de base pour
charger le programme \code{prog.hex} sur le PRisme est la suivante (à lancer depuis
le dossier dans lequel se trouve pygaload.py):
\begin{lstlisting}[style=console]
  ./pygaload.py prog.hex -p /dev/ttyUSB0 -V
\end{lstlisting}
Où \code{/dev/ttyUSB0} est le port sur lequel se trouve le programmateur USB.

Pour faciliter le téléchargement des programmes à l'avenir, il y a deux choses
que l'on peut faire. Tout d'abord, ajouter pygaload au \code{\$PATH} pour pouvoir
l'appeler depuis n'importe où.

\begin{lstlisting}
  cp pygaload.py /usr/local/bin/pygaload
\end{lstlisting}

On peut maintenant appeler pygaload depuis n'importe quel dossier en tapant
simplement \code{pygaload} au lieu de \code{./pygaload.py}.

Pour faciliter les choses encore plus, la meilleure option est d'ajouter une
entrée au Makefile qui appelle automatiquement pygaload avec les bons arguments
pour chaque projet. Le Makefile fourni avec le projet d'exemple contient déjà
une commande \code{flash} qui utilise pygaload pour charger le programme sur le
microcontrôleur. On peut donc charger le programme en invocant la commande
\code{make flash} depuis le dossier du projet.

\section{Makefile}
% TODO
blablabla

\section{Comment contribuer}
Si vous avez des idées d'améliorations à apporter à la librairie ou à la
documentation, vous pouvez contribuer vos modifications à travers GitHub. Si
nous pensons que les autres membres peuvent en profiter, nous les inclurons
(peut-être) dans les librairies Robopoly.

[ [ [ TODO : Très incomplet ! ] ] ] 

\subsection{Documentation}
Le présent document est écrit en \LaTeX{} . Vous pouvez sans autre télécharger la
source du document et y apporter des modifications.
% TODO add link to github, requirements for latex, etc

\subsection{Code C}

Marche à suivre :
\begin{list}{}{}
  \item Créer un compte sur GitHub
  \item Faire un fork
  \item Faire des modifications locales
  \item Push sur son repo public
  \item Faire une demande de pull
\end{list}

\begin{lstlisting}[style=console]
  git clone git://github.com/tunebird/robopoly-linux.git
\end{lstlisting}

\end{document}
